\documentclass{article}
\usepackage[utf8]{inputenc}
\usepackage{amsmath, amssymb}
\usepackage{tcolorbox}
\usepackage{multicol} % Para manejar múltiples columnas
\usepackage[a4paper, landscape, margin=1cm]{geometry} % Modo horizontal (landscape)

% Definir entorno para los cuadros con título negro
\newtcolorbox{teorema}[1]{
  colback=white,
  colframe=black,
  coltitle=white,
  colbacktitle=black,
  fonttitle=\bfseries,
  title={#1},
  boxrule=1pt,
  width=\linewidth,
  arc=2mm
}

\begin{document}

\begin{center}
    \LARGE \textbf{Campo eléctrico. Corriente eléctrica}
\end{center}

\begin{multicols}{3} % Dividimos en TRES columnas

% Primera columna
\begin{teorema}{Ley de Coulomb}
    $$\vec{F} = k\frac{|q_1q_2|}{r^2}[N] = k\frac{q_1q_2}{r^2}(\vec{r_1} - \vec{r_2}) [N]$$
    Cte de Coulomb $k = 8,99 \cdot 10^9 [N\cdot \frac{m^2}{C^2}]$\\
    Permitividad en el vacío $\epsilon_0 = 8,85 \cdot 10^-12 [\frac{F}{m}]$\\
    Permitividad en el medio $\epsilon = \epsilon_0\epsilon_r$
\end{teorema}

\begin{teorema}{Campo eléctrico}
    $$\vec{E} = k\frac{q_0}{r^2}\vec{r}\quad [\frac{N}{C}]$$
\end{teorema}

\begin{teorema}{Corriente Eléctrica}
    $$ I = \frac{Q}{t} \quad [A] = [\frac{1 C}{1 s}]$$
    Se suele dar en cables o movimientos de carga $\implies$\\
    $$ I = \frac{\partial{Q}}{\partial{t}} \text{ ó } I = \frac{\Delta{Q}}{\Delta{t}}$$
\end{teorema}


% Segunda columna
\columnbreak


\end{multicols} % Fin de columnas

\end{document}
