\usepackage{amsmath} % Para fórmulas matemáticas
\usepackage{amssymb} % Para símbolos matemáticos
\usepackage{xcolor} % Para definir colores
\usepackage[a4paper, margin=2cm]{geometry} % Ajuste de márgenes
\usepackage{tcolorbox} % Para cajas de colores
\usepackage{enumitem} % Para listas
\usepackage{tikz} % Para crear gráficos
\tikzset{>=latex} % for LaTeX arrow head
\usepackage{pgfplots}
\pgfplotsset{compat=1.18} % Versión más reciente de pgfplots (cámbialo a 1.13 si tienes problemas)
\usetikzlibrary{decorations.markings,intersections,calc,angles} % Librerías de TikZ
\usepackage{multicol} % Para columnas múltiples
\usepackage{pdflscape} % Para páginas en orientación horizontal

% Definir entorno para los cuadros con título negro
\newtcolorbox{teorema}[1]{
  colback=white,
  colframe=black,
  coltitle=white,
  colbacktitle=black,
  fonttitle=\bfseries,
  title={#1},
  boxrule=1pt,
  width=\linewidth,
  arc=2mm
}

\colorlet{Ecol}{orange!90!black}
\colorlet{Bcol}{blue!90!black}
\colorlet{veccol}{green!45!black}

\linespread{1.1} % Interlineado

\newenvironment{problem}[2][Ejercicio]{%
  \noindent\textbf{#1 #2.}%
}{\par}

\newenvironment{sol}{%
  \vspace{0.5em}\noindent\textit{Solución:}%
}{\hfill $\square$} % Se usa $\square$ en lugar de \qed por si no se usa amsthm