
\usepackage[utf8]{inputenc} % Codificación UTF-8
\usepackage{graphicx} % Para incluir imágenes
\usepackage{amsmath} % Para fórmulas matemáticas
\usepackage{amssymb} % Para símbolos matemáticos
\usepackage{physics} % Para enlaces
\usepackage{thmtools} % Para estilos de teoremas
\usepackage{amsthm} % Para estilos de teoremas
\usepackage{mdframed} % Para marcos
\usepackage{xcolor} % Para definir colores
\usepackage[a4paper, total={6.5in, 9.5in}]{geometry}
\usepackage{tikz}
\definecolor{LightRed}{rgb}{1.0, 0.8, 0.8} % Definir el color LightRed


\newtheorem{theorem}{Teorema} % Definir contador 'theorem' primero
\numberwithin{theorem}{subsection} % Numera por subsecciones


% Apply styles to the theorem
\usepackage{mdframed}
\colorlet{LightGreen}{green!15} % Define un color verde claro

\usepackage{mdframed}
\colorlet{LightGreen}{green!15} % Define un color verde claro

\newenvironment{proposicion}[1]{%
    \begin{mdframed}[backgroundcolor=LightGreen, 
        hidealllines=true, % Oculta los bordes
        innertopmargin=12pt, innerbottommargin=10pt, % Reduce los márgenes arriba/abajo
        skipbelow=5pt, skipabove=5pt, % Reduce el espacio fuera de la caja
        roundcorner=5pt]
    \textbf{#1}\\% Título en negrita
    \vspace{0.3em} % Espaciado entre el título y el contenido
}{%
    \end{mdframed}
}

