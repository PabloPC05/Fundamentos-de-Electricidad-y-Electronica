\usepackage[utf8]{inputenc} % Codificación UTF-8
\usepackage{graphicx} % Para incluir imágenes
\usepackage{amsmath} % Para fórmulas matemáticas
\usepackage{amssymb} % Para símbolos matemáticos
\usepackage{physics} % Para operadores matemáticos
\usepackage{amsthm} % Debe ir antes de thmtools
\usepackage{thmtools} % Para estilos de teoremas
\usepackage{mdframed} % Para marcos
\usepackage{xcolor} % Para definir colores
\usepackage[a4paper, total={6.5in, 9.5in}]{geometry}
\usepackage{tcolorbox} % Para cajas de colores
\usepackage{ulem} % Para subrayados
\usepackage{enumitem} % Para listas
\usepackage{hyperref} % Para enla
\usepackage{tikz} % Para crear gráficos
\tikzset{>=latex} % for LaTeX arrow head
%\usepackage{xcolor}
%\colorlet{force}{orange!80!black}
    \tikzstyle{charge}=[thin,top color=red!50,bottom color=red!70,shading angle=20]
    \tikzstyle{charge+}=[thin,top color=red!50,bottom color=red!90!black,shading angle=20]
    \tikzstyle{charge-}=[thin,top color=blue!50,bottom color=blue!80,shading angle=20]
    \tikzstyle{force}=[->,very thick,orange!80!black]
    \tikzstyle{vector}=[->,very thick,green!45!black]

    \def\F{1.8}\colorlet{LightGreen}{green!15} % Define un color verde claro
    \tikzset{>=latex} % for LaTeX arrow head
    \usepackage{pgfplots}
    \pgfplotsset{compat=1.13}
    \usetikzlibrary{decorations.markings,intersections,calc}
    \usepackage[outline]{contour} % glow around text
    \usetikzlibrary{angles,quotes} % for pic (angle labels)
    \colorlet{Ecol}{orange!90!black}
    \colorlet{EcolFL}{orange!80!black}
    \colorlet{Bcol}{blue!90!black}
    \tikzstyle{EcolEP}=[blue!80!white]
    \colorlet{veccol}{green!45!black}
    \tikzset{EFieldLineArrow/.style={EcolFL,decoration={markings,mark=at position #1 with {\arrow{latex}}},
                                    postaction={decorate}},
            EFieldLineArrow/.default=0.5}
 

\iffalse
\declaretheoremstyle[
    headfont  =\bfseries, % Ponemos el título en negrita
    notefont = \bfseries, 
    notebraces={}{},
    bodyfont=\normalfont,
    headpunct={},
    postheadspace =\newline, % Espacio tras el título
    headformat={\noindent #1\par}, % Alinea el título a la izquierda y salta de línea
    mdframed ={
        backgroundcolor=LightGreen, % Color de fondo
        hidealllines=true,          % Ocultar bordes
        innertopmargin=10pt, innerbottommargin=10pt, 
        innerleftmargin=10pt, innerrightmargin=10pt,
        roundcorner=5pt             % Bordes redondeados
    }
] {caja_verde}

\declaretheorem[
    style = caja_verde, 
    name = {}, 
    numbered = no
    ]
{proposicion}
\fi


% Definir un nuevo entorno para teoremas con un cuadro rojo claro
\newtcolorbox{teorema}[1]{colback=red!10!white, colframe=red!80!black, 
  title={\textbf{#1}}, fonttitle=\bfseries, after={\noindent\vspace{5pt}}}

  \linespread{1.1} % Interlineado