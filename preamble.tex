\usepackage[utf8]{inputenc} % Codificación UTF-8
\usepackage{graphicx} % Para incluir imágenes
\usepackage{amsmath} % Para fórmulas matemáticas
\usepackage{amssymb} % Para símbolos matemáticos
\usepackage{physics} % Para operadores matemáticos
\usepackage{amsthm} % Debe ir antes de thmtools
\usepackage{thmtools} % Para estilos de teoremas
\usepackage{mdframed} % Para marcos
\usepackage{xcolor} % Para definir colores
\usepackage[a4paper, total={6.5in, 9.5in}]{geometry}
\colorlet{LightGreen}{green!15} % Define un color verde claro

\declaretheoremstyle[
    headfont  =\bfseries, % Ponemos el título en negrita
    notefont = \bfseries, 
    notebraces={}{},
    bodyfont=\normalfont,
    headpunct={},
    postheadspace =\newline, % Espacio tras el título
    headformat={\noindent #1\par}, % Alinea el título a la izquierda y salta de línea
    mdframed ={
        backgroundcolor=LightGreen, % Color de fondo
        hidealllines=true,          % Ocultar bordes
        innertopmargin=10pt, innerbottommargin=10pt, 
        innerleftmargin=10pt, innerrightmargin=10pt,
        roundcorner=5pt             % Bordes redondeados
    }
] {caja_verde}

\declaretheorem[
    style = caja_verde, 
    name = {}, 
    numbered = no
    ]
{proposicion}