\section{Campo eléctrico. Corriente eléctrica.}
\subsection{Carga eléctrica}
La \textbf{carga eléctrica} es una propiedad de la materia que le permite interactuar con \textbf{campos eléctricos}.\\ Puede ser positiva o negativa y determina como los objetos interactúan entre sí mediante fuerzas eléctricas.

\begin{proposicion}{Principio de atracción y repulsión}
    Las cargas de igual signo se repelen y las cargas de signo contrario se atraen.
\end{proposicion}

La carga es una magnitud cuantificada y su unidad de medida es el \textbf{Coulomb} (C). La carga de un electrón es de $-1.6 \times 10^{-19}$ C y la de un protón es de $1.6 \times 10^{-19}$ C.
\begin{proposicion}{Conservación de la carga}
    La carga eléctrica no se crea ni se destruye, solo se transforma.
\end{proposicion}

    